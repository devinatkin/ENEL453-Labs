\subsubsection{Create Top Level}

With the test bench and this module complete you have the minimum needed number of lower level cells. You could decide to make a specialized cell for clock division, or you can reuse the PWM module with appropriate values to produce a slow clock. You need this slower clock to drive your circular shift register so that you can visualize the patterns being created. You can use the provided PWM test bench to find a good maximum value, and duty cycle through trial and error or deliberate calculation. \\
\vspace{0.5cm}
For the top level:
\begin{itemize}
    \item Instantiate 16 copies of the PWM module. Hook one PWM module up to each LED output. 
    \item Instantiate 1 copy of your circular shift register. Hook each PWM duty cycle up to one of the output registers. 
    \item Instantiate either an appropriately configured PWM module, or a dedicated clock divider to produce a slow clock for the circular shift register. 
    \item Modify the provided top level test bench and produce the needed text file to run through the python web application to verify your results visually. 
\end{itemize}
Extend the LED Output on the top module.
\begin{verbatim}
    output reg [15:0] led
\end{verbatim}
Declare 16 copies of the PWM Module
\begin{verbatim}
    pwm_module #(
        .bit_width(bit_width)
    ) pwm_unit (
        .clk(clk),
        .rst_n(rst_n),
        .duty(duty[i]),
        .max_value(8'd255),
        .pwm_out(pwm_out[i])
    );
\end{verbatim}
For these modules we are going to want a bit width of 8 bits to match with the circular shift register we created, and then we are going to set our maximum value to 255. duty[i] is going to be outputs from the circular shift register and will be 0 to 15. We will need to declare the associated logic to hook up from the pwm modules to the circular shift register. Finally pwm\_out is going to be directed towards the 16 led outputs. \\

Let's declare the variables to hook everything together.
\begin{verbatim}
    logic rst_n;
    logic [7:0] duty [15:0];
    logic [15:0] pwm_out;
    // Output of the circular shift register
    logic [7:0] reg_out [15:0]; 
    // Reduced clock signal for the circular shift register
    logic clk_reduced; 
\end{verbatim}

You can see here I have declared rst\_n. This is because I have added this line to the top module.
\begin{verbatim}
    assign rst_n = ~reset;
\end{verbatim}
This is technically breaking the rule of no logic on the top level of a design. But this can be considered the absolute most that is acceptable, even if it is frowned upon. If you want to adhere to absolute best practices, create an additional module which will take a reset input, and deliver rst, and rst\_n outputs. These outputs may be generated through either combinational or sequential logic. In larger designs appropriate handling of reset can become a significant challenge. 

\begin{verbatim}
    circular_shift_register csr (
        .clk(clk_reduced),
        .rst_n(rst_n),
        .reg_out(reg_out)
    );
\end{verbatim}

Declare your circular shift register. You'll notice that I'm feeding it a reduced clock speed. If cycling at the 100MHz clock cycle it would loop around every 160ns, even if the pwm module were running fast enough to reflect this it is far too fast for our eyes to see. Next you'll want to declare an additional PWM module to generate your output values. 

\begin{verbatim}
        pwm_module #(
        .bit_width(?) 
    ) clk_reducer (
        .clk(clk),
        .rst_n(rst_n),
        .duty(?), 
        .max_value(?),
        .pwm_out(clk_reduced)
    );
\end{verbatim}

Find appropriate values for duty, max\_value, and bit\_width to create a visible cycling of the led brightness.\\
\vspace{1cm}
Lastly \textbf{remember} to add appropriate assigns to connect the circular shift register to the pwm modules, and the pwm outputs to the leds. With this complete create a new top level test bench to verify the behavior. 

\subsubsection{Top Level Test bench}
You can reuse the old top level test bench. With a few minor changes. 
\begin{itemize}
    \item Remove the switch adjusting behavior and simply wait a long period of time after the reset is complete.
    \item Modify the print to file function call to pass all the led values. Removing the zeros and replacing them with the led values. \begin{verbatim}
        print_leds_to_file({led,15'b000000000000000}, file_descriptor);
    \end{verbatim}
\end{itemize}

Run your simulation again, this will take 5 - 20 minutes. Afterwards your led.txt will contain lines like this:
\begin{verbatim}
    0.0000150000 0000000000000000
    0.0000450000 1111111111111111
    0.0000550000 0000011111111000
    0.0002150000 0000001111110000
    0.0003750000 0000000111100000
    0.0006950000 0000000011000000
\end{verbatim}
The timing will be different and will shift as you progress through the very long file. You can then produce a video representation to confirm you have a good led pattern using the same online tool as for the original design. 

\begin{figure}[H]
    \centering
    \includegraphics[height=1cm]{Images/ProvidedPWM/FinalOutput.jpg}
    \caption{Generated Video Output}
    \label{fig:enter-label}
\end{figure}