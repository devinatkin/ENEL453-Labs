PWM Standards for Pulse Width Modulation. This means that the duty cycle of a square wave is being changed over time, this can be simply to dimm an led or for exciting reasons such as communicating information.\\ 
\vspace{0.5cm}
\begin{tikztimingtable}
PWM Signal 0\% Duty Cycle  & LLLLLL \\
PWM Signal 25\% Duty Cycle & LHLLLH \\
PWM Signal 50\% Duty Cycle & LHHLLH \\
PWM Signal 75\% Duty Cycle & LHHHLH \\
PWM Signal 100\% Duty Cycle & HHHHHH \\
  % \extracode
  % \draw (0,0) circle(0.2pt); % Origin Dot
  % \begin{pgfonlayer}{background}
  %   \vertlines[help lines] {0.5,4.5}
  % \end{pgfonlayer}
\end{tikztimingtable}

\vspace{0.5cm}

Here the duty cycle is changing between 4 different states where 0\% represents the output being fully off, and 100\% represents the output being fully on, and the duty cycles in-between represent being on for a different percentage of the time. The easiest way to create this sort of PWM behavior is to have a counter which increments after each clock cycle until hitting a maximum value and resetting. The duty cycle is set by a number, the output is then 1 while the number is less than or equal to it, and zero when it is above it. 
