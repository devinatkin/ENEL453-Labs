This is an examination TAs are not allowed to help you, you are guiding them through the project, explaining the reasoning behind your design. 

\begin{itemize}
    \item The Design is presented clearly and cleanly with a logical process being followed. Marks will be deducted for rambling explanations or unclear design structure. A good presentation will be concise and present a vision of the overall system architecture with explanations of the different major design blocks as well as how they relate to one another. \begin{flushright}
        \framebox[1.1\width]{\underline{\hspace{1cm}}\textbackslash2\%}
    \end{flushright}
    \item The Design is fully functional with no apparent glitches in the design behavior. There is a creative element present which goes beyond the elements used in the earlier labs. If needed the creative component can be demonstrated separately, if you are not able to fully integrate the creative elements; however, this will result in some mark deduction. \begin{flushright}
        \framebox[1.1\width]{\underline{\hspace{1cm}}\textbackslash3\%}
    \end{flushright}
    \item The student can clearly explain the functionality of any block components in their design. The student can explain how the specific element relates to the overall design and how the element functions internally. The student can explain how the design element may be reused. \begin{flushright}
        \framebox[1.1\width]{\underline{\hspace{1cm}}\textbackslash5\%}
    \end{flushright}
    \item The student can explain the functionality of a piece of code in their design when asked, explaining how it fits into the larger context of the design. If there are multiple different approaches the student should explain why they took that specific approach.\begin{flushright}
        \framebox[1.1\width]{\underline{\hspace{1cm}}\textbackslash5\%}
    \end{flushright}
\end{itemize}

\vspace{1cm}

\subsubsection{How the Oral Examination will be Run}

\begin{itemize}
    \item The TA will be setup at a table with Vivado loaded up. When you have submitted the design you can ask the TA to complete the examination. They will download your design files onto the computer, and allow you to load them into a Vivado Project for the lab. 
    \item You will have 5 minutes to walk the TA through your design. This can be done on a white board, through a presentation, or through walking them through the Vivado project. This presentation should take no more than 5 minutes. You will be cut-off if you go over time.\\
    \item The TA will then ask the student to load their design onto the Basys 3 board.
    \item The TA will spend 1-2 minutes going through the design functionality on the board. The student can explain their button and switch assignments as well as their creative element.
    \item The TA will then ask the student to explain a component or multiple components, chosen by the TA on their design block diagram. The student will need to explain the functionality of the component and how it works in the broader context of the design.
    \item The TA will then go to a random section of the students code and ask them to explain how it functions. If the code is extensively documented it is valid for the student to read the comments written in the code to job their memory and help explain the functionality. 

    \item Students will not receive their mark on the oral exam until after the examinations are complete across all participants.
\end{itemize}

