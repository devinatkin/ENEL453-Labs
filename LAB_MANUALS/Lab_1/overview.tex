
\normalsize
\ifdefined\devinisteaching
ENEL 453 was one of my favorite courses when I took it during my undergrad. It helped inspire me towards pursuing VLSI as a field, and gained me a substantial leg up when it came to designing FPGA based systems. The reason it was as effective as it was is because I played relentlessly with the code, improving both my test benches and my designs. I encourage you to do the same, play with your designs and go beyond what the labs ask of you, I assure you it is the best way to do well in this course. 
\else
 ENEL 453 is a crucial course that can shape your understanding and enthusiasm for VLSI and FPGA-based systems. The effectiveness of this course comes from deeply engaging with the code, refining test benches, and pushing the boundaries of the lab requirements. Engaging in this manner is highly recommended for succeeding in this course.
\fi
\bigskip

\ifdefined\devinisteaching
A word of warning to students who scoff at test benches. By the end of this course your synthesis times for your FPGA may end up exceeding 1 hour. Effective test benches are the only way to complete your labs in a reasonable length of time. If you go further you may end up with synthesis times taking days to complete. Learn to make quality test benches early, and you'll save more time in the long run. 
\else
A word of warning to students who scoff at test benches. By the end of this course your synthesis times for your FPGA may end up exceeding 1 hour. Effective test benches are the only way to complete your labs in a reasonable length of time. If you go further you may end up with synthesis times taking days to complete. Learn to make quality test benches early, and you'll save more time in the long run. 
\fi
\bigskip

\ifdefined\devinisteaching
I am happy to say we are moving to one of my personal favorite FPGA boards. The Basys3 by digilent, a nifty little board with lots 
of peripherals, and a surprisingly powerful Artix 7 FPGA. The manual for the board is available here: \url{\basys3manual}. 
\else
  This course will use the Basys3 by Digilent, an excellent FPGA board that comes with a variety of peripherals and features a powerful Artix 7 FPGA. The manual for the board is available here: \url{\basys3manual}.
\fi